% Created 2022-12-28 Wed 19:02
% Intended LaTeX compiler: pdflatex
\documentclass[11pt]{article}
\usepackage[utf8]{inputenc}
\usepackage[T1]{fontenc}
\usepackage{graphicx}
\usepackage{longtable}
\usepackage{wrapfig}
\usepackage{rotating}
\usepackage[normalem]{ulem}
\usepackage{amsmath}
\usepackage{amssymb}
\usepackage{capt-of}
\usepackage{hyperref}
\author{Biggus Diccus}
\date{\today}
\title{}
\hypersetup{
 pdfauthor={Biggus Diccus},
 pdftitle={},
 pdfkeywords={},
 pdfsubject={},
 pdfcreator={Emacs 28.1 (Org mode 9.5.5)}, 
 pdflang={English}}
\begin{document}

\tableofcontents

\section{Analisi 2}
\label{sec:orga13deeb}
\subsection{Non ricordo}
\label{sec:orgb940c26}
\begin{itemize}
\item forma quadratica, perché è imporante la matrice simmetrica?
\item importanza delle matrici simmetriche (se la cosa non riguardava Schwarz)
\end{itemize}

\subsection{Limiti}
\label{sec:org343cbde}
\begin{itemize}
\item teorema di permanenza del segno
\item lemma di collegamento
\begin{itemize}
\item per evitare le cazzate tipo quelle che fai col seno e i limiti
\item \(\lim_{x \to x_0} f(x) = L \iff\) per ogni
successione \(a_k \to x_0\) \(lim{k \to \infty} f(a_k) = L\)
\end{itemize}
in poggiolinese
\begin{description}
\item[{ipotesi}] \begin{itemize}
\item \(E \subseteq \mathbb{R}^n, f : E \to \mathbb{R}\)
\item \(x_0 \in E\)
\item \(L \in \mathbb{R}\)
\end{itemize}
\item[{allora}] \[
     \lim_{x \to x_0} f(x) = L \iff
     \forall (a_k)_{k \in \mathbb{N}}
     \text{ con } a_k \in E \forall k \text{ e }
     \lim_{k \to \infty} a_k = x_0 \text{ allora }
     \lim_{k \to \infty} f(a_k) = L
     \]
\end{description}
\end{itemize}

\subsection{Curve}
\label{sec:org0a3db65}
\begin{itemize}
\item cosa si intende per curva parametrica
\item cos'è \(\gamma'(t_0)\) (derivata di una curva parametrica)
\item definire la lunghezza di una curva
\item limite di curva
\item parametrizzare la circonferenza interna
\item lunghezza di un arco di curva
\item parametro d'arco
\item equivalenza di due curve
\item integrale di linea
\end{itemize}

\subsubsection{Curve regolari}
\label{sec:org41208d9}
\begin{itemize}
\item definizione curva regolare
\item definizione di curva, tutte le proprietà di continuità e regolarità di una curva
\end{itemize}

\subsection{Massimi e minimi}
\label{sec:org2c92753}
\begin{itemize}
\item studio dei massimi e minimi assoluti di una funzione
\item data una \(f : (A \subseteq \mathbb{R}^2) \to \mathbb{R}\), cosa vuol dire che \(x_0\) è un
punto di massimo locale?
\item teorema di Fermat, dimostrazione
\item Se \(f\) derivabile infinite volte, come faccio a trovare i candidati a essere
massimi/minimi locali (vedi punti critici, Fermat)
\item \(f : \mathbb{R}^2 \to \mathbb{R}\), come trovo massimi e minimi assoluti di \(f\) in \(E\)?
\end{itemize}

\subsection{Cazzate di funzione}
\label{sec:org1f15983}
\begin{itemize}
\item se ho una funzione misurabile non negativa, posso scriverla sempre come liimite di una
seccessione di funzioni più semplici?
\end{itemize}

\subsection{Derivate}
\label{sec:org21f4db1}
\begin{itemize}
\item definizione di derivata parziale
\begin{itemize}
\item fissi tutte le variabili apparte \emph{quella} e fai il limite al tendere di \emph{quella}
\end{itemize}
\item come si garantisce la continuità?
\begin{itemize}
\item derivabile può non essere continua, ma puoi garantirlo se è differenziabile
\end{itemize}
\item derivata di \((f \circ \gamma) (t)\) (credo che
\(f : \mathbb{R^n} \to \mathbb{R}\) e
\(\gamma : \mathbb{R} \to \mathbb{R}^2\))
\item \$\(\gamma\)(t) = \{x = t, y = t\textsuperscript{3}\}, trova \(\gamma '\)
\item derivata direzionale e formula del gradiente
\item teorema di Shwarz
\begin{itemize}
\item cambiando l'ordine delle variabili la derivata non cambia
\item[{ipotesi}] \begin{itemize}
\item \(A\) aperto \(\subseteq \mathbb{R}^n\), e \(x_0 \in A\)
\item \(f : A \to \mathbb{R}\)
\item \(i \neq j\) e
\(\frac{\partial}{\partial x_i}\frac{\partial f}{\partial x_j},
         \frac{\partial}{\partial x_j}\frac{\partial f}{\partial x_i}\)
esistono continue in un intorno/palla di \(x_0\) con raggio \(> 0\)
\end{itemize}
\item[{allora}] \[
     \frac{\partial}{\partial x_i} \frac{\partial f}{\partial x_j} =
     \frac{\partial}{\partial x_j} \frac{\partial f}{\partial x_i}
     \]
\end{itemize}
\item importanza delle matrici simmetriche (Shwartz, fai metà della fatica? Resto boh)
\item definizione matrice hessiana
(quella con tutte le derivate seconde)
\end{itemize}

\subsubsection{Derivabilità}
\label{sec:orgb5a8823}
\begin{itemize}
\item esempio di funzione derivabile ma non continua
\item data \(f : (A \subseteq \mathbb{R}^2) \to \mathbb{R}\), concetto di derivabilità e gradiente
\item \(f : \mathbb{R}^2 \to \mathbb{R}\), cosa vuol dire che \(f\) derivabile in
\((x_0, y_0) \in \mathbb{R}^2\)
\end{itemize}

\subsubsection{Punti critici}
\label{sec:orgb463bdf}
\begin{itemize}
\item definizione punto critico, perchè sono importanti
\item punto di sella?
\end{itemize}

\subsection{Differenziale \& Co.}
\label{sec:org2bbf26a}
\begin{itemize}
\item definizione di differenziabile \\
puoi usare il gradiente per fare Taylor e funziona effettivamente come approssimazione
quindi localmente è piatta e si comporta bene, vale a dire che non tende a 20 valori
contemporaneamente, in poggioninese
\end{itemize}


\begin{itemize}
\item dimostrare che se \(f\) differenziabile allora \(f\) continua
\item dimostrare che se \(f\) differenziabile allora \(f\) derivabile
\item \(f : \mathbb{R}^2 \to \mathbb{R}\), cosa vuol dire che \(f\) differenziabile in
\((x_0, y_0) \in \mathbb{R}^2\)
\item condizioni sufficienti per la differenziabilità
\end{itemize}

\subsection{Coordinate polari}
\label{sec:org25f91b5}
\begin{itemize}
\item \(C = {(x,y) \in \mathbb{R}^2 | x^2 + y^2 \leq 4\), scrivi in forma polare
\item dato \(C\) della domanda di sopra, impostare \(\int_C \sqrt{x^2 + y^2}\)
\item \(E = {(x,y) \in \mathbb{R}^2 | y > x - 1, 1 \leq {(x-1)}^2 + y^2 \leq 4}\)
riscrivilo in coordinate polari\footnote{poi riscrivilo in rust}, e imposta di una generica
funzione, dare una caratterizzazione dell'insieme \(E\)
\end{itemize}

\subsection{Lesbegue}
\label{sec:orgfbbd7cc}
\begin{itemize}
\item definizione di funzione semplice
(una combinazione lineare finita di altipiani, se fai infiniti alitpiani puoi far
tendere altre funzioni)
\item definizione integrale di Lesbegue
\item funzione misurabile secondo Lesbegue
\item continuità della misura, con dimostrazione
\item n-intervallo, volume, e misura esterna
\item proprietà della misura esterna
\item proprietà degli insiemi misurabili secondo Lesbegue
\item proprietà delle funzioni misurabili
\item fetta di un insieme, teorema di Fubini
\item Beppo Levi, con dimostrazione
\end{itemize}

\subsection{Puta madre}
\label{sec:orgdaa59d1}
\begin{itemize}
\item \(E = {(x,y) \in \mathbb{R}^2 | x^2 + y^2 < 4, x^2 - {(y-1)}^2 > 1}\)
disegnarlo e studiarlo
\item \(E = {(x,y) \in \mathbb{R}^2 | y \leq x, {(x-1)}^2 + y^2 \leq 1 }\)
disegnarlo e farci integrale
\item \(E = {(x,y) \in \mathbb{R}^2 | y \geq x, 1 \leq x^2 + y^2 \leq 4}\)
disegnarlo
\item \(E = {(x,y) \in \mathbb{R}^2 | x^2 - y \leq 4, {(x-1)}^2 + y^2 \leq 4}\)
disegna questa merda e imposta un integrale su \(E\)
\end{itemize}

\section{Probabilità}
\label{sec:orgab16de0}
\end{document}